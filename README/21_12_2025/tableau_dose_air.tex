% Tableau des résultats de simulation Geant4
% Configuration : AIR (sans blindage), 1×10^6 événements
% Source : Eu-152 (44 kBq), Détecteur : Sphère d'eau (r=2 cm) à 18 cm

\documentclass[11pt,a4paper]{article}
\usepackage[utf8]{inputenc}
\usepackage[T1]{fontenc}
\usepackage[french]{babel}
\usepackage{booktabs}
\usepackage{siunitx}
\usepackage{array}
\usepackage{multirow}
\usepackage{caption}

\sisetup{
    separate-uncertainty = true,
    multi-part-units = single,
    per-mode = symbol
}

\begin{document}

% ============================================================================
% TABLEAU PRINCIPAL : Résultats des trois méthodes - Configuration AIR
% ============================================================================

\begin{table}[htbp]
\centering
\caption{Débits de dose calculés par les trois méthodes de simulation Geant4 -- Configuration AIR (sans blindage)}
\label{tab:dose_air}
\begin{tabular}{@{}lccc@{}}
\toprule
& \textbf{Méthode 1} & \textbf{Méthode 1bis} & \textbf{Méthode 2} \\
& \textbf{(MC direct)} & \textbf{(Forçage)} & \textbf{(Fluence $\times \mu_{en}/\rho$)} \\
\midrule
\multicolumn{4}{@{}l}{\textit{Statistiques de détection}} \\[2pt]
\quad Événements avec dépôt & 7\,069 & --- & --- \\
\quad Gammas traversant détecteur & --- & 23\,612 & 23\,612 \\
\quad Énergie totale (keV) & 893\,204 & 946\,664 & 946\,664 \\
\quad Énergie/gamma généré (keV/$\gamma$) & 0.465 & 0.492 & 0.492 \\
\midrule
\multicolumn{4}{@{}l}{\textit{Débits de dose}} \\[2pt]
\quad Débit brut (nGy/h) & 676.4 & 716.9 & 716.9 \\[3pt]
\quad \textbf{Débit corrigé (nGy/h)} & $\mathbf{169.1 \pm 4.2}$ & $\mathbf{179.2 \pm 1.2}$ & $\mathbf{179.2 \pm 1.2}$ \\
\midrule
\multicolumn{4}{@{}l}{\textit{Comparaison avec la valeur théorique (174.8 nGy/h)}} \\[2pt]
\quad Écart relatif (\%) & 3.3 & 2.5 & 2.5 \\
\quad Incertitude statistique (\%) & 2.5 & 0.65 & 0.65 \\
\bottomrule
\end{tabular}
\end{table}

% ============================================================================
% TABLEAU COMPACT : Résumé pour publication
% ============================================================================

\begin{table}[htbp]
\centering
\caption{Résumé des débits de dose simulés -- Configuration AIR}
\label{tab:resume_air}
\begin{tabular}{@{}lc@{}}
\toprule
\textbf{Méthode} & \textbf{Débit de dose (nGy/h)} \\
\midrule
Valeur théorique & 174.8 \\
\midrule
Méthode 1 -- Dépôt d'énergie MC & $169.1 \pm 4.2$ \\
Méthode 1bis -- Forçage d'interaction & $179.2 \pm 1.2$ \\
Méthode 2 -- Fluence $\times \mu_{en}/\rho$ & $179.2 \pm 1.2$ \\
\bottomrule
\end{tabular}
\end{table}

% ============================================================================
% TABLEAU DES PARAMÈTRES DE SIMULATION
% ============================================================================

\begin{table}[htbp]
\centering
\caption{Paramètres de la simulation Geant4 -- Configuration AIR}
\label{tab:parametres_air}
\begin{tabular}{@{}ll@{}}
\toprule
\textbf{Paramètre} & \textbf{Valeur} \\
\midrule
\multicolumn{2}{@{}l}{\textit{Source}} \\
\quad Radionucléide & Eu-152 \\
\quad Activité ($4\pi$) & \SI{44}{\kilo\becquerel} \\
\quad Gammas moyens/désintégration & 1.924 \\
\midrule
\multicolumn{2}{@{}l}{\textit{Simulation}} \\
\quad Nombre d'événements & $1 \times 10^6$ \\
\quad Gammas générés & 1\,922\,741 \\
\quad Demi-angle du cône d'émission & \SI{60}{\degree} \\
\quad Facteur de correction $f_{\text{cône}}$ & 0.25 \\
\quad Temps simulé & \SI{22.73}{\second} \\
\midrule
\multicolumn{2}{@{}l}{\textit{Géométrie}} \\
\quad Distance source--détecteur & \SI{18}{\centi\meter} \\
\quad Rayon du détecteur (eau) & \SI{2}{\centi\meter} \\
\quad Masse du détecteur & \SI{33.51}{\gram} \\
\quad Configuration & AIR (sans blindage) \\
\midrule
\multicolumn{2}{@{}l}{\textit{Transmission}} \\
\quad Gammas transmis & 97.9\% \\
\quad Gammas absorbés & 2.0\% \\
\bottomrule
\end{tabular}
\end{table}

\end{document}
